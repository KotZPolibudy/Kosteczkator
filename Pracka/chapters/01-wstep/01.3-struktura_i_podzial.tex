\section{Struktura pracy oraz podział obowiązków}\label{sec:struktura-pracy-oraz-podzia-obowiazkow}

Struktura pracy jest następująca:
\begin{itemize}
    \item Rozdział~\ref{ch:przeglad-rynku} przedstawia przegląd obecnie dostępnych sprzętowych generatorów liczb losowych,
    w tym liderów na rynku oraz innych rozwiązań, zbliżonych do naszego.
    \item Rozdział~\ref{ch:budowa-sprzetowego-generatora} jest poświęcony budowie sprzętowej części robota.
    \item Rozdział~\ref{ch:odczytywanie-losowego-wyniku-z-kosci} zawiera dokładny opis metody odczytywania wyniku rzutu kością,
    wraz z opisem wykorzystanej sieci neuronowej.
    \item Rozdział~\ref{ch:komunikacja} opisuje protokół komunikacji z innym urządzeniem odbierającym wynik pracy robota.
    \item Rozdział~\ref{ch:testy} przedstawia wykorzystane testy statystyczne oraz ich wyniki.
    \item Rozdział~\ref{ch:zakonczenie} zawiera ostateczne wnioski.
\end{itemize}

%---- TODO ---- \\
%ej musimy we wstępie gdzieś wcisnąć ze einstein tez uważał ze rzuty kością są losowe bo powiedział Bog nie rzuca kośćmi \\
%~ Jakub Kędra

Realizacja projektu została podzielona w sposób przedstawiony w tabeli~\ref{tab:podzial}.
\begin{table} [h]
    \centering
    \caption{Podział obowiązków w projekcie}
    \label{tab:podzial}
    \begin{tabular}{|l|l|}
        \hline
        Imię i nazwisko & Zakres pracy \\
        \hline
        \multirow{9}{*}{Jakub Kędra} 
            & Projekt i wydruk w technologii druku 3D prototypów oraz ostatecznej \\
            & wersji urządzenia. \\
            & Projekt układu elektronicznego. \\
            & Montaż elementów strukuralnych, połączenie elementów elektrycznych \\
            & i elektronicznych. \\
            & Implementacja oprogramowania sterującego komponentami urządzenia. \\
            & Oznaczanie przykładów uczących dla sieci neuronowej. \\
            & Wykonanie testów sprawdzających poprawność działania urządzenia. \\
            & Zaprojektowanie przepływu pracy. \\
            & Przegląd literatury. \\
        \hline
        \multirow{11}{*}{Wojciech Kot} 
            & Projekt i implementacja części związanej z odczytem i przetwarzaniem \\
            & zdjęć z kamery. \\
            & Projekt i implementacja sieci neuronowej dokonującej klasyfikacji \\
            & wyniku rzutu kością. \\
            & Przygotowanie zbiorów treningowych i testowych do trenowania sieci \\
            & neuronowej. \\
            & Wykonanie testów sprawdzających poprawność działania urządzenia. \\
            & Oznaczanie przykładów uczących dla sieci neuronowej. \\
            & Zaprojektowanie przepływu pracy. \\
            & Główny pomysłodawca projektu. \\
            & Przegląd literatury. \\
        \hline
        \multirow{7}{*}{Jakub Prusak} 
            & Zaprojektowanie i implementacja interfejsu komunikacji i sterowania \\
            & urządzeniem\\
            & Przygotowanie komunikacji maszyny z innymi urządzeniami. \\
            & Zrównoleglenie przetwarzania. \\
            & Wykonanie testów sprawdzających poprawność działania urządzenia. \\
            & Oznaczanie przykładów uczących dla sieci neuronowej. \\
            & Przegląd literatury. \\
        \hline
        \multirow{7}{*}{Julia Samp} 
            & Opracowanie i przeprowadzenie testów statystycznych otrzymanych \\
            & wyników. \\
            & Implementacja testów zgodnie z standardem FIPS. \\
            & Porównanie i interpretacja wyników otrzymanych dla obu modeli. \\
            & Wykonanie testów sprawdzających poprawność działania urządzenia. \\
            & Oznaczanie przykładów uczących dla sieci neuronowej. \\
            & Korekta tekstu. \\
            & Przegląd literatury. \\
        \hline  
    \end{tabular} 
\end{table}   
