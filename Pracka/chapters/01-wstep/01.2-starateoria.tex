Jedną z najważniejszych decyzji w trakcie realizacji projektu był wybór rodzaju kości, jaką miał odczytywać
i interpretować model sztucznej inteligencji.
Jako najistotniejsze kryterium wyznaczono liczbę ścian będącą potęgą liczby dwa
(już na wstępnym etapie projektowania odrzucając standardową kość sześciościenną) --
w celu łatwej interpretacji wyniku w notacji binarnej, stosowanej powszechnie w informatyce, w tym -- kryptografii.
W następnej kolejności szukano kompromisu pomiędzy łatwością w odczycie ścianek a generowaniem jak największej liczby
bitów jednym rzutem -- czyli kości o jak największej liczbie ścianek.

Z dostępnych powszechnie na rynku kości tylko cztero-, ośmio- i szesnastościenne spełniają pierwszy z wymienionych wyżej kryteriów.
Kość czterościenną odrzucono początkowo ze względu na jej kształt ostrosłupa foremnego o podstawie trójkąta,
na której liczby zapisywane są na rogach, a nie ściankach (co pokazano na rysunku~\ref{fig:k4}),
jednak powróciła ona do dyskusji w postaci niestandardowych kształtów kości, które pokazano na rysunku~\ref{fig:nietypowe_modern_k4}.
Niestety nie na długo, z racji na bardzo niską dostępność na runku tego typu kości, oraz rzucające się w oczy problemy
z odczytywaniem pustych (zaokrąglonych) ścianek kości typu modern, pokazanych na rysunku~\ref{fig:modern_k4},
oraz takich z bardzo małymi oznaczeniami, które przestałyby być widoczne przy nawet niewielkim przechyleniu kości
dla ściętego ostrosłupa pokazanych na rysunku~\ref{fig:nietypowe_k4}.

Kość szesnastościenną również odrzucono ze względu na mały rozmiar ścianek oraz stosunkowo niewielki kąt nachylenia ścianek względem siebie,
który sprawia, że na zdjęciu robionym idealnie nad kością widoczne jest kilka ścianek jednocześnie, jak pokazano na rysunku~\ref{fig:k16}.
Z pozostałych opcji tylko na kości ośmiościennej (pokazanej na rysunku~\ref{fig:k8}) widać z góry dokładnie jedną ściankę,
a do tego ma liczbę ścianek będącą potęgą 2, zatem to ten rodzaj kości został wybrany do użycia w projekcie.

\begin{figure}[h]
    \centering
      \begin{subfigure}{.3\textwidth}
        \includegraphics[width=.9\linewidth, trim={250mm 200mm 350mm 150mm}, clip]{chapters/02-teoria/figures/k4}
        \caption{\label{fig:k4}Kość czterościenna}
      \end{subfigure}%
      \begin{subfigure}{.3\textwidth}
        \includegraphics[width=.9\linewidth, trim={250mm 200mm 350mm 150mm}, clip]{chapters/02-teoria/figures/k8}
        \caption{\label{fig:k8}Kość ośmiościenna}
      \end{subfigure}%
       \begin{subfigure}{.3\textwidth}
        \includegraphics[width=.9\textwidth, trim={300mm 150mm 300mm 200mm}, clip]{chapters/02-teoria/figures/k16}
        \caption{\label{fig:k16}Kość szesnastościenna}
      \end{subfigure}
    \caption{Zdjęcia kości z góry}
\end{figure}

\begin{figure}[h]
    \centering
      \begin{subfigure}{.45\textwidth}
        \includegraphics[width=.9\linewidth]{chapters/02-teoria/figures/modern_k4}
        \caption{\label{fig:modern_k4}Kość czworościenna typu „modern”}
      \end{subfigure}%
      \begin{subfigure}{.45\textwidth}
        \includegraphics[width=.9\linewidth, trim={250mm 200mm 350mm 150mm}, clip]{chapters/02-teoria/figures/nietypowe_k4}
        \caption{\label{fig:nietypowe_k4}Kość czworościenna ze ściętym czubkiem}
      \end{subfigure}%
    \caption{Nietypowe kości czworościenne}
    \label{fig:nietypowe_modern_k4}
\end{figure}