\chapter{Wstęp}\label{ch:wstep}

%Wstęp do pracy powinien zawierać następujące elementy:
%\begin{itemize}
%    \item krótkie uzasadnienie podjęcia tematu;
%    \item cel pracy (patrz niżej);
%    \item zakres (przedmiotowy, podmiotowy, czasowy) wyjaśniający, w jakim rozmiarze praca będzie realizowana;
%    \item ewentualne hipotezy, które autor zamierza sprawdzić lub udowodnić;
%    \item krótką charakterystykę źródeł, zwłaszcza literaturowych;
%    \item układ pracy (patrz niżej), czyli zwięzłą charakterystykę zawartości poszczególnych rozdziałów;
%    \item ewentualne uwagi dotyczące realizacji tematu pracy np.~trudności, które pojawiły się w trakcie
%    realizacji poszczególnych zadań, uwagi dotyczące wykorzystywanego sprzętu, współpraca z firmami zewnętrznymi.
%\end{itemize}

% \noindent
% \textbf{Wstęp do pracy musi się kończyć dwoma następującymi akapitami:}
% \begin{quote}
% Celem pracy jest opracowanie / wykonanie analizy / zaprojektowanie / ...........
% \end{quote}
% oraz:
% \begin{quote}
% Struktura pracy jest następująca. W rozdziale 2 przedstawiono przegląd literatury na temat ........
% Rozdział 3 jest poświęcony ....... (kilka zdań).
% Rozdział 4 zawiera ..... (kilka zdań) ............ itd.
% Rozdział X stanowi podsumowanie pracy.
% \end{quote}
%
% W przypadku prac inżynierskich zespołowych lub magisterskich 2-osobowych, po tych dwóch w/w akapitach
% musi w pracy znaleźć się akapit, w którym będzie opisany udział w pracy poszczególnych członków zespołu. Na przykład:
%
% \begin{quote}
% Jan Kowalski w ramach niniejszej pracy wykonał projekt tego i tego, opracował ......
% Grzegorz Brzęczyszczykiewicz wykonał ......, itd.
% \end{quote}

%---Koniec Template---

Współczesne systemy komputerowe i technologie wykorzystują liczne algorytmy generowania liczb losowych, które stanowią podstawę w takich dziedzinach jak kryptografia,
symulacje komputerowe czy gry komputerowe.
Jednakże większość tych rozwiązań opiera się na generatorach pseudolosowych,
które, mimo swojej wydajności, w istocie generują liczby deterministyczne.
W praktyce może to prowadzić do problemów z bezpieczeństwem i nieprzewidywalnością w sytuacjach, gdzie prawdziwa losowość jest kluczowa.

Inspiracją do podjęcia tematu budowy sprzętowego generatora liczb losowych było połączenie zainteresowań w dziedzinie techniki oraz pasji do gier RPG i planszowych.
Gry te, od wieków korzystające z kostek jako narzędzia generowania losowości, stanowią idealny przykład zastosowania fizycznych mechanizmów do uzyskiwania nieprzewidywalnych wyników.
Projekt bazuje na hipotezie, że poprzez zautomatyzowanie procesu rzutu kością możliwe jest uzyskanie prawdziwej losowości, mimo deterministycznego charakteru działania maszyny.

Celem pracy jest zbudowanie urządzenia zdolnego do generowania liczb losowych poprzez mechaniczny rzut kością oraz analiza stopnia losowości uzyskanych wyników.
Projekt ten zakłada stworzenie urządzenia, które będzie w stanie rzucać kością w sposób powtarzalny i przewidywalny,
jednak z losowym efektem wynikającym z natury procesów fizycznych, takich jak tarcie, turbulencje czy drgania.
Dodatkowo praca ma na celu ocenę możliwości praktycznego zastosowania takiego generatora jako źródła entropii w systemie Linux.

Realizacja takiego urządzenia pozwoli nam także w prosty sposób zautomatyzować statystyczne badanie różnych
kości do gry pod względem sprawdzania ich potencjalnej nieuczciwości,
co było wyjściowym pomysłem od którego dopiero powstała idea stworzenia HTRNG opartego na rzucie kością.

Realizacja tego projektu pozwoli nie tylko na sprawdzenie skuteczności podejścia opartego o rzuty kośćmi,
ale również na pogłębienie wiedzy o granicach między deterministycznymi systemami a losowością,
stanowiąc wkład w rozwój technologii oraz praktycznych zastosowań w obszarach wymagających prawdziwej losowości.



