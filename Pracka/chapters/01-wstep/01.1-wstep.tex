\chapter{Wstęp}\label{ch:wstep}

%todo ale boga czy Boga, oraz czy to dopisywać?
%W swoim liście z 1926r. Albert Einstein pisał do Maxa Borna podkreślając swą wiarę w deterministyczny świat.
%\textit{„Ja w każdym razie jestem głęboko przeświadczony, że On nie gra w kości.”}
%Dla komputerów również wszystko jest deterministyczne, a jednak korzystamy z losowości przy tak wielu operacjach... bla bla, dobra, Kuba sam se wstaw ten cytat bo ja bardzo nie umiem tego ubrać w słowa

Współczesne systemy komputerowe i technologie wykorzystują liczne algorytmy generowania liczb losowych, które stanowią podstawę w takich dziedzinach jak kryptografia,
symulacje komputerowe czy gry komputerowe.
Jednakże większość tych rozwiązań opiera się na generatorach pseudolosowych,
które, mimo swojej wydajności, w istocie generują liczby deterministyczne.
W praktyce może to prowadzić do problemów z bezpieczeństwem i nieprzewidywalnością w sytuacjach, gdzie prawdziwa losowość jest kluczowa.

Inspiracją do podjęcia tematu budowy sprzętowego generatora liczb losowych (ang. HRNG -- Hardware Random Number Generator)
było połączenie zainteresowań w dziedzinie techniki oraz pasji do gier RPG i planszowych.
Gry te, od wieków korzystające z kostek jako narzędzia generowania losowości, stanowią idealny przykład zastosowania fizycznych mechanizmów do uzyskiwania nieprzewidywalnych wyników.
Projekt bazuje na hipotezie, że poprzez zautomatyzowanie procesu rzutu kością możliwe jest uzyskanie prawdziwej losowości, mimo deterministycznego charakteru działania maszyny.

Celem pracy jest zbudowanie urządzenia zdolnego do generowania liczb losowych poprzez mechaniczny rzut kością oraz analiza stopnia losowości uzyskanych wyników.
Projekt ten zakłada stworzenie urządzenia, które będzie w stanie rzucać kością w sposób powtarzalny i przewidywalny,
jednak z losowym efektem wynikającym z natury procesów fizycznych, takich jak tarcie, turbulencje czy drgania.
Dodatkowo praca ma na celu praktyczne zastosowanie stworzonego generatora jako źródła entropii w systemie Linux.

Realizacja takiego urządzenia pozwoli nam także w prosty sposób zautomatyzować statystyczne badanie różnych
kości do gry pod względem sprawdzania ich potencjalnej nieuczciwości,
co było wyjściowym pomysłem, od którego dopiero powstała idea stworzenia sprzętowego generatora liczb prawdziwie losowych (ang.
HTRNG -- Hardware True Random Number Generator) opartego na rzucie kością.

Realizacja tego projektu pozwoli nie tylko na sprawdzenie skuteczności podejścia opartego o rzuty kośćmi,
ale również na pogłębienie wiedzy o granicach między deterministycznymi systemami a losowością,
stanowiąc wkład w rozwój technologii oraz praktycznych zastosowań w obszarach wymagających prawdziwej losowości.

