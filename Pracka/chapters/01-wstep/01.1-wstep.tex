\chapter{Wstęp}\label{ch:wstep}

%W swoim liście z 1926r. Albert Einstein pisał do Maxa Borna podkreślając swą wiarę w deterministyczny świat.
%\textit{„Ja w każdym razie jestem głęboko przeświadczony, że On nie gra w kości.”}
%Dla komputerów również wszystko jest deterministyczne, a jednak korzystamy z losowości przy tak wielu operacjach... bla bla, dobra, Kuba sam se wstaw ten cytat bo ja bardzo nie umiem tego ubrać w słowa

%ej musimy we wstępie gdzieś wcisnąć ze einstein tez uważał ze rzuty kością są losowe bo powiedział Bog nie rzuca kośćmi \\
%~ Jakub Kędra

Współczesne systemy komputerowe i technologie wykorzystują liczne algorytmy generowania liczb losowych, które stanowią podstawę w takich dziedzinach jak kryptografia,
symulacje komputerowe czy gry komputerowe.
Jednakże większość tych rozwiązań opiera się na generatorach pseudolosowych,
które, mimo swojej wydajności, w istocie generują liczby deterministyczne.
W praktyce może to prowadzić do problemów z bezpieczeństwem i przewidywalnością w sytuacjach, gdzie prawdziwa losowość jest kluczowa.

Inspiracją do podjęcia tematu budowy sprzętowego generatora liczb losowych (HRNG, z ang. \textit{Hardware Random Number Generator})
było połączenie zainteresowań w dziedzinie techniki oraz pasji do gier planszowych i RPG. %czy należy wytłumaczyć rpg? XD
Gry te, korzystające od wieków z kości jako narzędzia generowania losowości, stanowią idealny przykład zastosowania fizycznych mechanizmów do uzyskiwania nieprzewidywalnych wyników.
Projekt stara się potwierdzić hipotezę, że poprzez zautomatyzowanie procesu rzutu kością możliwe jest uzyskanie prawdziwej losowości mimo deterministycznego charakteru działania maszyny.

\section{Cele i założenia projektu}\label{sec:cel-projektu}

Celem pracy jest zbudowanie urządzenia zdolnego do generowania liczb losowych poprzez mechaniczny rzut kością oraz analiza stopnia losowości uzyskanych wyników.
Projekt ten zakłada wytworzenie urządzenia, które będzie w stanie rzucać kością w sposób powtarzalny i przewidywalny,
jednak z losowym efektem wynikającym z natury procesów fizycznych, takich jak tarcie, turbulencje czy drgania.
Dodatkowo praca ma na celu praktyczne zastosowanie zbudowanego generatora jako źródła entropii w systemie Linux.

Realizacja takiego urządzenia pozwoli także w prosty sposób zautomatyzować statystyczne badanie różnych
kości do gry pod względem sprawdzania ich potencjalnej nieuczciwości,
co było wyjściowym pomysłem, od którego dopiero powstała idea zbudowania sprzętowego generatora liczb prawdziwie losowych
(HTRNG, z ang. \textit{Hardware True Random Number Generator}) opartego na rzucie kością.

Realizacja tego projektu pozwoli nie tylko na sprawdzenie skuteczności podejścia opartego o rzuty kośćmi,
ale również na pogłębienie wiedzy o granicach między deterministycznymi systemami a losowością,
stanowiąc wkład w rozwój technologii oraz praktycznych zastosowań w obszarach wymagających prawdziwej losowości.

Projekt robota został opracowany w oparciu o kilka kluczowych założeń, które mają na celu zapewnienie jego funkcjonalności, efektywności oraz użyteczności:
\begin{enumerate}
    \item Założenie niewielkich rozmiarów i kompaktowości.
    \item Założenie modułowości elementów robota.
    \item Założenie prostej budowy.
    \item Założenie możliwie szybkiego uzyskiwania wyniku.
    \item Założenie możliwie niskich kosztów projektu.
\end{enumerate}

Niewielkie rozmiary robota są kluczowe, ponieważ pozwalają na łatwe wykorzystanie urządzenia nawet w codziennym życiu.
Kompaktowa konstrukcja oznacza mniejszą wagę i większą mobilność, przez co robota można bez problemu przenosić w różne miejsca.
Dzięki temu robot jest bardziej uniwersalny w zastosowaniu i może być używany w różnych celach.

Modułowość pozwala na łatwą rozbudowę i modyfikację robota.
Poszczególne komponenty mogą być wymieniane, naprawiane lub ulepszane bez konieczności ingerowania w całą konstrukcję.
Dzięki temu projekt jest bardziej elastyczny, co zwiększa jego żywotność i możliwość dostosowania do nowych zastosowań.
Modułowość ułatwia również składanie, rozkładanie oraz serwisowanie urządzenia.

Prosta konstrukcja zapewnia, że robot jest łatwy do złożenia, obsługi i serwisowania.
Minimalizuje to ryzyko występowania błędów podczas montażu oraz potencjalne problemy eksploatacyjne.
Prosta budowa sprawia, że robot jest łatwiejszy w obsłudze dla osób mniej zaawansowanych technicznie, co zwiększa dostępność
urządzenia dla szerokiego grona użytkowników, co ma znaczenie w przypadku komercjalizacji rozwiązania.
Jednocześnie upraszcza to proces projektowania, co pozwala na szybsze przejście od koncepcji do gotowego produktu.

Szybkość działania robota jest kluczowa dla zwiększenia jego efektywności.
Urządzenie, które w szybkim czasie może uzyskać rezultat jest bardziej użyteczne niż
takie, które na uzyskanie wyniku potrzebuje długiego czasu. 

Optymalizacja kosztów konstrukcji i implementacji sprawia, że projekt jest bardziej dostępny dla potencjalnych użytkowników i może być wykorzystywany w praktyce a nie tylko w warunkach laboratoryjnych. 
Dobór odpowiednich komponentów i technologii pozwala na obniżenie ogólnych wydatków bez utraty jakości.
Dzięki temu realizacja staje się możliwa nawet przy ograniczonym budżecie.

Realizacja projektu w oparciu o powyższe założenia umożliwia zaprojektowanie robota, który jest nie tylko
i wydajny, ale również przystępny w produkcji i użytkowaniu.
