
\chapter{Wstęp}

Wstęp do pracy powinien zawierać następujące elementy:
\begin{itemize}
    \item krótkie uzasadnienie podjęcia tematu; 
    \item cel pracy (patrz niżej); 
    \item zakres (przedmiotowy, podmiotowy, czasowy) wyjaśniający, w jakim rozmiarze praca będzie realizowana; 
    \item ewentualne hipotezy, które autor zamierza sprawdzić lub udowodnić; 
    \item krótką charakterystykę źródeł, zwłaszcza literaturowych; 
    \item układ pracy (patrz niżej), czyli zwięzłą charakterystykę zawartości poszczególnych rozdziałów; 
    \item ewentualne uwagi dotyczące realizacji tematu pracy np.~trudności, które pojawiły się w trakcie 
    realizacji poszczególnych zadań, uwagi dotyczące wykorzystywanego sprzętu, współpraca z firmami zewnętrznymi. 
\end{itemize}

\noindent
\textbf{Wstęp do pracy musi się kończyć dwoma następującymi akapitami:}
\begin{quote}
Celem pracy jest opracowanie / wykonanie analizy / zaprojektowanie / ...........
\end{quote}
oraz:
\begin{quote}
Struktura pracy jest następująca. W rozdziale 2 przedstawiono przegląd literatury na temat ........ 
Rozdział 3 jest poświęcony ....... (kilka zdań). 
Rozdział 4 zawiera ..... (kilka zdań) ............ itd. 
Rozdział X stanowi podsumowanie pracy. 
\end{quote}

W przypadku prac inżynierskich zespołowych lub magisterskich 2-osobowych, po tych dwóch w/w akapitach 
musi w pracy znaleźć się akapit, w którym będzie opisany udział w pracy poszczególnych członków zespołu. Na przykład:

\begin{quote}
Jan Kowalski w ramach niniejszej pracy wykonał projekt tego i tego, opracował ......
Grzegorz Brzęczyszczykiewicz wykonał ......, itd. 
\end{quote}

