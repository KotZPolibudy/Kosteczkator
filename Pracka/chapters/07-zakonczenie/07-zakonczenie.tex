
\chapter{Zakończenie}\label{ch:zakonczenie}

W ramach realizacji projektu udało się spełnić jego cele, jakim było stworzenie sprzętowego generatora liczb losowych.
Wykorzystuje on rzut kością, a wynik jest odczytywany i interpretowany przez model sztucznej inteligencji.
Przeprowadzono testy statystyczne, badające losowość otrzymanych wyników.
Pokazały one, że możliwe jest uzyskanie prawdziwej losowości zautomatyzowanego rzutu kością,
mimo deterministycznego działania wykonującej go maszyny.


%todo nazwać to lepiej i przepisać, bo obecnie to draft!!
\section{Plany na przyszłość]}
Mimo osiągnięcia celów projektu, możliwe jest wprowadzenie wielu usprawnień, które poprawią jego jakość i wygodę użytkownika.
Korzystnym ulepszeniem byłoby zintegrowanie maszyny nie tylko z systemem Linux, ale również z popularnym systemem Windows.
Dodatkowo, można również zwrócić uwagę na obecnie używane metody przetwarzania zdjęć, oraz czy nie jest możliwe uzyskanie „czystszych” zdjęć wejściowych dla sieci neuronowej, np.
poprzez dokładniejsze wycięcie tła, lub poprzez może lepszy hardware (kamerę o małej ogniskowej, lepsze oświetlenie)
Rozwiązaniem, które na pewno zachęciłoby wielu fanów gier planszowych i RPG, byłoby rozszerzenie funkcjonalności robota do
zbierania statystyk dowolnej kości, a więc rozszerzenie modelu sztucznej inteligencji o odczytywanie kości w innych wzorach i kolorach,
a także o innych ilościach ścianek.
W ten sposób można by wykryć potencjalne nieprawidłowości w budowie sprawdzanej kości, które mogą zaburzać rozkład generowanych przez nią wyników.

