
\chapter{Zakończenie}\label{ch:zakonczenie}

W ramach realizacji projektu udało się spełnić jego główny cel, jakim było stworzenie sprzętowego generatora liczb losowych.
Wykorzystuje on rzut kością, a wynik jest odczytywany i interpretowany przez model sztucznej inteligencji. Przeprowadzono
testy statystyczne, badające losowość otrzymanych wyników. Pokazały one, że możliwe jest uzyskanie prawdziwej losowości 
zautomatyzowanego rzutu kością, mimo deterministycznego działania wykonującej go maszyny.

Mimo osiągnięcia celów projektu, możliwe jest wprowadzenie wielu usprawnien, które poprawią jego jakość i wygodę użytkownika.
Korzystnym ulepszeniem byłoby zintegrowanie maszyny nie tylko z systemem Linux, ale również z popularnym systemem
Windows. 

Rozwiązaniem, które na pewno zachęciłoby wielu fanów gier planszowych i RPG, byłoby rozszerzenie modelu 
sztucznej inteligencji o odczytywanie kości w innych wzorach i kolorach, a także o innych ilościach ścianek.
Wprowadziwszy te ulepszenia, możliwe będzie również zbieranie statystyk używanej kości. Pozwoli to na wykrycie 
potencjalnych nieprawidłowości w jej budowie, które mogą zaburzać rozkład generowanych przez nią wyników.

Dodatkowo, można również usprawnić wykorzystywane metody przetwarzania zdjęć oraz uzyskać 
„czystsze” zdjęcia wejściowe dla sieci neuronowej. Wśród rozważanych sposobów na uzyskanie tego efektu znajdują się
dokładniejsze wycięcie tła czy poprawienie hardware, np. wprowadzając kamerę o małej ogniskowej lub 
lepsze oświetlenie dna kubeczka.

Jednym z najambitniejszych do zaimplementowania usprawnien jest nauczenie modelu sztucznej inteligencji odczytywania 
wyników kilku kości na raz. Pozwoliłoby to na generowanie kilkukrotnie większej liczby bitów w tym samym czasie. Dzięki
temu udałoby się choć częściowo zmniejszyć największą wadę zrealizowanego projektu, jakim jest wolne generowanie nowych
liczb.

Wśród innych planów na najbliższą przyszczłość znajduje się również dodanie praktycznego zastosowania dla guzika robota,
której nie udało się zaimplementować wcześniej przez ogarniczenia czasowe. Wśród rozważanych funkcjonalności znajduje się
przełączanie między trybami robota (opisanymi w rozdziale \ref{sec:tryby}) bądź wykonywanie pojedynczych rzutów kością.

Kolejnym z planowanych ulepszeń jest wymiana komputera Raspberry Pi na tańszy model. Ponieważ rozważana jest komercjalizacja
produktu, taka zmiana pozwoliłaby obniżyć jego cenę. Sama maszyna mogłaby znaleźć zastosowanie w kryptografii, gdzie 
szybkość generowania nowych liczb nie jest kluczowa. Byłaby również ciekawym gadżetem dla fanów gier planszowych i RPG,
oraz oferować rzuty kością dla takich gier online.

To jedynie część z pomysłów na dalsze rozwijanie projektu, które narodziły się w trakcie kolejnych etapów jego tworzenia.
Zrealizowała go czteroosobowa grupa, dzięki owocnej współpracy i jasnemu podziałowi zadań. 
Finealnie, projekt został ukończony z pełnym sukcesem, a wszystkie jego założenia zostały zrealizowane. 