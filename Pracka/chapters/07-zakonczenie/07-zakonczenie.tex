
\chapter{Zakończenie}\label{ch:zakonczenie}

W ramach realizacji projektu udało się spełnić jego główny cel, jakim było stworzenie sprzętowego generatora liczb losowych.
Wykorzystuje on rzut kością, a wynik jest odczytywany i interpretowany przez model sztucznej inteligencji. Przeprowadzono
testy statystyczne, badające losowość otrzymanych wyników. Pokazały one, że możliwe jest uzyskanie prawdziwej losowości 
zautomatyzowanego rzutu kością, mimo deterministycznego działania wykonującej go maszyny. 
Sumarycznie, w czasie realizacji projektu, robot wykonał ponad 30 000 rzutów kością.

Mimo osiągnięcia celów projektu, możliwe jest wprowadzenie wielu usprawnień, które poprawią jego jakość i wygodę użytkownika.
Korzystnym ulepszeniem byłoby zintegrowanie maszyny nie tylko z systemem Linux, ale również z popularnym systemem Windows.

Rozwiązaniem, które na pewno zachęciłoby wielu fanów gier planszowych i RPG, byłoby rozszerzenie modelu 
sztucznej inteligencji o odczytywanie kości w innych wzorach i kolorach, a także o innych ilościach ścianek.
Wprowadziwszy te ulepszenia, możliwe będzie zbieranie statystyk używanej kości, co pozwoli na wykrycie
potencjalnych nieprawidłowości w jej budowie, które mogą zaburzać rozkład generowanych przez nią wyników.

Co więcej, można również usprawnić wykorzystywane metody przetwarzania zdjęć oraz uzyskać
„czystsze” zdjęcia wejściowe dla sieci neuronowej. Wśród rozważanych sposobów na uzyskanie tego efektu znajdują się
dokładniejsze wycięcie tła czy poprawienie hardware poprzez np. wprowadzenie kamery o małej ogniskowej lub
lepsze oświetlenie dna kubka.

Innym ulepszeniem, które zwiększyłoby możliwości robota jest nauczenie modelu sztucznej inteligencji odczytywania
wyników kilku kości jednocześnie lub zaimplementowanie przetwarzania, które pozwoliłoby na odczytywanie wyników z dwóch kości.
Pozwoliłoby to na generowanie kilkukrotnie większej liczby bitów w tym samym czasie.
Dzięki temu udałoby się chociaż częściowo zmniejszyć największą wadę zrealizowanego projektu,
jaką jest wolne generowanie nowych liczb.
Jednakże to usprawnienie wiąże się z trudnością, wykonywania samego rzutu kilkoma kośćmi czy innmi problemami takimi jak kości potencjalnie 
upadające na sobie, uniemożliwiając efektywny odczyt.

Wśród innych planów na najbliższą przyszłość znajduje się również dodanie zastosowania dla przycisku robota. Wśród rozważanych funkcjonalności znajduje się
przełączanie między trybami robota (opisanymi w rozdziale~\ref{sec:tryby}) bądź wykonywanie pojedynczych rzutów kością.

W przypadku komercjalizacji produktu należy rozważyć również wymianę komputera Raspberry Pi na tańszy model lub stworzenie dedykowanego rozwiązania.
Taka zmiana pozwoliłaby obniżyć cenę komercjalizowanego produktu, a sama maszyna mogłaby znaleźć zastosowanie nie tylko w kryptografii,
ale również wśród fanów gier planszowych oraz RPG, gdzie mogłaby znaleźć zastosowanie jako urządzenie generujące sprawiedliwe rzuty kością
dla graczy chcących grać zdalnie za pośrednictwem sieci.

To jedynie część z pomysłów na dalsze rozwijanie projektu, które narodziły się w trakcie kolejnych etapów jego realizacji.
Wykonała go czteroosobowa grupa, dzięki owocnej współpracy i jasnemu podziałowi zadań. 
Finalnie, projekt został ukończony z pełnym sukcesem, a wszystkie jego założenia zostały zrealizowane.

W pracy zostały wykorzystane generatywne modele SI, przy sprawdzaniu poprawności składniowej i językowej zdań oraz generowaniu drobnych fragmentów wykorzystanego kodu.

%przy tym projekcie nie ucierpiała żadna kostka,
%jeden silnik,
%wykonano pierdyliard rzutów