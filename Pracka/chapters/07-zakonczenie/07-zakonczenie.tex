
\chapter{Zakończenie}\label{ch:zakonczenie}

W ramach realizacji projektu udało się spełnić jego cele, jakim było stworzenie sprzętowego generatora liczb losowych.
Wykorzystuje on rzut kością, a wynik jest odczytywany i interpretowany przez model sztucznej inteligencji. Przeprowadzono
testy statystyczne, badające losowość otrzymanych wyników. Pokazały one, że możliwe jest uzyskanie prawdziwej losowości 
zautomatyzowanego rzutu kością, mimo deterministycznego działania wykonującej go maszyny.

Mimo osiągnięcia celów projektu, możliwe jest wprowadzenie wielu usprawnien, które poprawią jego jakość i wygodę użytkownika.
Korzystnym ulepszeniem byłoby zintegrowanie maszyny nie tylko z systemem Linux, ale również z popularnym systemem
Windows. Rozwiązaniem, które na pewno zachęciłoby wielu fanów gier planszowych i RPG, byłoby rozszerzenie modelu 
sztucznej inteligencji o odczytywanie kości w innych wzorach i kolorach, a także o innych ilościach ścianek.

