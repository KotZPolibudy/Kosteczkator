
\chapter{Budowa sprzętowego generatora liczb losowych}
section{Projektowanie robota}

Proces projektowania robota rozpoczęto od przeanalizowania różnych metod wykonywania rzutu kością.
Ostatecznie, po przeanalizowaniu kilku koncepcji, zdecydowano się na rozwiązanie wykorzystujące obrotowy 
kubek, wewnątrz którego kość porusza się i odbija od ścianek. Taki mechanizm zapewnia losowość rzutu, a 
jednocześnie jest prosty w konstrukcji i intuicyjny w działaniu. Obracający się kubek został 
zaprojektowany tak, aby jego prędkość i czas trwania obrotu można było precyzyjnie kontrolować, co  w założeniu 
pozwala na uzyskanie wiarygodnych wyników przy każdym rzucie. W celach testowych został skonstruowany prototypowy
model robota, zbudowany w taki sposób, żeby wszystkie jego komponenty były modułowe. Takie rozwiązanie pozwala na 
łatwą wymianę elementów robota, bez potrzeby przeprojektowywania całego robota od nowa. Przy budowie wykorzystano 
technologię druku 3D, która pozwala na szybkie modyfikacje przy jednoczesnym zachowaniu bardzo wysokiej dokładności
budowy elementów składowych.

Pierwszy prototyp robota składał się z metalowych prętów służących za stelaż oraz elementów wydrukowanych na drukarce 3D.
Tymi elementami był: kubek, ramię służące do montażu kubka, uchwyty do prątów oraz płytka mocująca do kamery. Dodatkowo
wykorzystano silnik prądu stałego napędzający kubek oraz sterownik służący do zasilania i sterowania ruchem silnika.

\begin{figure}[H]
    \centering
    \includegraphics[width=0.25\linewidth]{chapters/03-praca-wlasna/figures/pierwszy}
    \caption{\label{fig:pierwszy}bałagan na stole}
\end{figure}
    

Po pierwszych testach okazało się, że niezbędny do uzyskania zamierzonego efektu będzie mechanizm, który będzie 
wychylał cały kubek wraz z silnikiem, który odpowiada za jego obrót. Z początku planowano wykorzystanie prostogo
serwomechanizmu jednak to rozwiązanie odrzucono, ponieważ większość dostępnych serwomechanizmów, które byłyby odpowiednie w tym celu ma
ograniczony ruch do $180^{\circ}$ lub $360^{\circ}$ a to limitowałoby możliwości mechanizmu służącego do wychylania kubka.
Ostatecznie w tym celu wybrano mały silnik krokowy z wystarczającym momentem obrotowym (34.3mN.m). Silnik ten obraca 
układem dwóch kół zębatych 1:2 dzięki czemu silnik ma jeszcze większy zapas momentu obrotowego. Dzięki takiemu rozwiązaniu silnik
nie pracuje na granicy swoich możliwości co zapewni jego długi okres eksploatacji. 

\begin{figure}[H]
    \centering
    \includegraphics[width=0.25\linewidth]{chapters/03-praca-wlasna/figures/koła_zębowe.png}
    \caption{\label{fig:zebatki}zebatki}
\end{figure}

Podczas testów pierwszej wersji robota wykorzytującej obrotowy kubek powstał pomysł alternatywnego rozwiązania.
Rozwiązanie to implementuje inne podejście do rzutu kością. Zamiast obracać cały kubek, a dodatkowo wyhylać go,
wykorzystany został trwale zamontowany kubek, na którego dnie znajduje się śmigło, które podcina leżącą na dnie kość.
Takie rozwiązanie znacząco upraszcza cały mechanizm robota oraz bardzo przyspiesza proces losowania liczby.

Przy projektowaniu drugiego wariantu robota został wykorzystany ten sam stelaż złożony z metalowych prętów co w 
pierwszym wariancie. Na drukarce 3D wydrukowano dodatkowe części, niezbędne do realizacji tego wariantu.
Zaprojektowano i wydrukowano nowy kubek, śmigło oraz mocowanie dla silnika. Kubek został przystosowany do montażu 
silnika prądu stałego oraz śmigła. 

%tu kłamie bo jeszcze nie zrobilem tego
%nie prawda, zrobiłeś
W obu wariantach dużym problemem był słaby obraz z kamery. W tym celu zaprojektowano system oświetlenia składający się z diod
LED, sterowanych za pomocą układu ULN2803A Darlington. Dzięki temu wnętrze kubka stało się dużo jaśniejsze, co pozwala kamerze na
robienie zdjęć o wystarczająco dobrej jakości dla zamierzonego celu. 

%TODO
%wstawic zdjeicei oswietlonego kubka

Duże znaczenie ma również wykorzystywana kość. Od jej koloru i tekstury zależy jakość zdjęć zrobionych przez
zamonotowaną kamerę. Poniżej przedstawiono dwa przykłady zdjęć i różnic w ich czytelności zależnych od koloru kości.

\begin{figure}[H]
    \centering
    \includegraphics[width=0.25\linewidth]{chapters/03-praca-wlasna/figures/nie_widac.png}
    \caption{\label{fig:nie_widac}gorzej}
\end{figure}

\begin{figure}[H]
    \centering
    \includegraphics[width=0.25\linewidth]{chapters/03-praca-wlasna/figures/widac.png}
    \caption{\label{fig:widac}lepiej}
\end{figure}

W trakcie testów zauważono, że procesor robota nagrzewa się do wysokich temperatur podczas intensywnej pracy, 
co mogło negatywnie wpływać na jego wydajność i żywotność. Aby temu zapobiec, w projekcie zdecydowano się na 
zastosowanie dodatkowych rezystorów, które miały pomóc w rozproszeniu nadmiaru ciepła, oraz wentylatora, który 
wspomagał cyrkulację powietrza wokół procesora. Dzięki temu rozwiązaniu udało się obniżyć temperaturę pracy 
procesora, co zapewniło stabilne i bezpieczne działanie całego systemu.

\begin{figure}[H]
    \centering
    \includegraphics[width=0.25\linewidth]{chapters/03-praca-wlasna/figures/now_we_are_cool.png}
    \caption{\label{fig:zimno}zimno}
\end{figure}

W obu wariantach dużym problemem był słaby obraz z kamery. W tym celu zaprojektowano system oświetlenia składający się z diod
LED sterowanych za pomocą układu ULN2803A Darlington. Dzięki temu wnętrze kubka stało się dużo jaśniejsze co pozwala kamerze na
robienie zdjęć o wystarczająco dobrej jakości dla zamierzonego celu. Dodatkowo rozświetlenie wnętrza kubka na tyle poprawiło
jakość zdjęć, że pozwoliło to na obniżenie kamery względem kubka. Dzięki temu wysokość prototypu zmniejszyła się o około 5cm.
Diody połączono szeregowo dzięki czemu nie trzeba było wykorzystywać dodatkowych rezystorów a ilość połączeń
została minimalna.

\begin{figure}[H]
    \centering
    \rotatebox{90}{\includegraphics[width=0.25\linewidth]{chapters/03-praca-wlasna/figures/i_stala_sie_jasnosc.jpg}}
    \caption{\label{fig:jasno}jasno}
\end{figure}

%%%%%%%%%%%%%%%%%%%%%%%%%%%%%%%%
%TODO
%1. oświetlenie 
%2. dodać wiecej informacji
%3. ostateczny projekt robota
%4. całe mnóstwo poprawiania i dopracowywania tego tekstu
%5. dodać zdjęcia/modele
