\section{Rozpoznawanie liczb}\label{sec:funkcja-predict}

W celu rozpoznawania liczb na zdjęciach przetworzonych przez model napisano funkcję \texttt{predict\_number},
która przekształca surowy obraz pobrany wcześniej z kamery do odpowiedniego formatu i zwraca przewidywaną klasę.
Funkcja działa w następujących krokach, opisanych w kolejnych podrozdziałach:
\begin{enumerate}
    \item Wczytanie i przetworzenie obrazu.
    \item Klasyfikacja obrazu (odczytanie wyniku).
    \item Interpretacja wyniku.
\end{enumerate}

\subsection{Wczytanie i przetworzenie obrazu}\label{subsec:wczytanie-i-przetworzenie-obrazu}

W przypadku funkcji dokonującej klasyfikacji na pliku, obraz należy najpierw wczytać.
Krok ten jest pomijany, gdy do funkcji zostanie przekazany jako parametr obraz wczytany wcześniej.
Obraz wejściowy następnie jest poddawany takim samym transformacjom, jak przy obrazach, na których trenowany był model,
a więc skalowany jest na rozmiar zgodny z wymaganiami modelu ($64 \times 64$ pikseli), zmieniany w skalę szarości, oraz normalizowany.

\subsection{Klasyfikacja obrazu}\label{subsec:klasyfikacja-obrazu}

Przygotowany obraz jest przekazywany do modelu w celu uzyskania wyników klasyfikacji:

\begin{verbatim}
p = model.predict(obraz_wejściowy)
klasyfikacja = np.argmax(p, axis=1)
\end{verbatim}

Funkcja \texttt{np.argmax} zwraca indeks klasy o najwyższym prawdopodobieństwie.

\subsection{Interpretacja wyniku}\label{subsec:interpretacja-wyniku}

Na podstawie indeksu przewidywanej klasy przypisywana jest odpowiadająca mu etykieta (od 1 do 8).
Wynikiem funkcji jest etykieta odpowiadająca numerowi na kości,
co umożliwia dalsze wykorzystanie tego w systemie.

W dalszym etapie przetwarzania, aby przejść z cyfr w systemie dziesiętnym na ich zapis binarny, wykonywana jest specyficzna transformacja.
Zakres $[1, 8]$ da się zapisać za pomocą trzech bitów, co w pełni wykorzystuje dostępny zakres (ta właściwość była również kluczową przesłanką w wyborze kości o dokładnie ośmiu ściankach).
Zdecydowano się interpretować ściankę oznaczoną numerem 8 jako cyfrę 0, co oznacza, że w transformacji na zapis binarny,
liczba 8 jest interpretowane jako 000, a nie 1000 jak mogłoby się intuicyjnie wydawać.
Finalnie, zaetykietowany wynik przetwarzany jest na ciąg trzech bitów, odpowiadających klasie.
