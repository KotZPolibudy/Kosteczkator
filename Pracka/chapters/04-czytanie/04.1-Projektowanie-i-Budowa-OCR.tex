%%%%%%%%%%%%%%%%%%%%%%%%%%%%%%%%
%TODO
%1. TODO - fotki, ktore juz sa na repo
%2. TODO preprocess nie kodem a fotkami i opisem
%3.
%4.

\chapter{Odczytywanie losowego wyniku z kości}

\section{Przetwarzanie wstępne obrazów}

W niniejszym rozdziale omówiono proces przetwarzania wstępnego obrazów kości, który przekształca dane pochodzące z fizycznego komponentu naszej maszyny (zdjęcia kości) na dane wejściowe dla modelu sztucznej inteligencji.
Przez dane wejściowe dla modelu SI rozumie się tutaj odpowiednio sformatowane obrazy, a więc takie w skali szarości, o rozmiarach 64x64 piksele, zawierające jedynie kość wyciętą ze zdjęcia całego bębna maszyny.

\subsection{algorytm}
Przedstawiony algorytm został zaimplementowany w języku Python, a jego zadaniem jest identyfikacja, wycięcie i przeskalowanie obszarów zawierających obiekty zainteresowania w zdjęciach.

Przetwarzanie obrazów składa się z kilku etapów, które dokładnie opisano poniżej, a workflow prezentuje się tak:
Obraz wejściowy -> Odnalezienie kości za pomocą maski na komponencie nasycenia -> Stworzenie i wycięcie "Bounding Box" wokół maski -> Przeskalowanie do odpowiedniego rozmiaru -> Konwersja do skali szarości -> Zapisanie gotowego obrazu


Zdjęcia w formacie JPEG są wczytywane za pomocą biblioteki Pillow, a następnie konwertowane do przestrzeni kolorów RGB, co zapewnia jednolitość formatów danych wejściowych:
\begin{verbatim}
image = Image.open(image_path).convert("RGB")
\end{verbatim}

Obrazy są przekształcane do przestrzeni barw HSV, aby oddzielić komponenty odpowiadające za barwę (H), nasycenie (S) oraz jasność (V).
Nasycenie jest następnie wygładzane przy użyciu filtra Gaussa:

\begin{verbatim}
hsv_image = image.convert("HSV")
h, s, v = hsv_image.split()
blurred_v = s.filter(ImageFilter.GaussianBlur(radius=5))
\end{verbatim}

Na podstawie komponentu nasycenia (Saturation) oraz progowania tworzona jest maska binarna, która identyfikuje obszary o wysokim nasyceniu (a więc naszą kość):

W celu usunięcia niewielkich luk w masce binarnej stosowana jest operacja zamknięcia morfologicznego z wykorzystaniem strukturalnego elementu prostokątnego:


TODO - cytowanie

\begin{verbatim}
kernel = cv2.getStructuringElement(cv2.MORPH_RECT, (100, 100))
closed_mask = cv2.morphologyEx(binary_mask, cv2.MORPH_CLOSE, kernel)
\end{verbatim}

Maska jest analizowana w celu zlokalizowania największego konturu, który określa obszar zainteresowania. Na tej podstawie oryginalny obraz jest kadrowany, a następnie przeskalowywany do wymiarów $64 \times 64$ pikseli:

\begin{verbatim}
contours, _ = cv2.findContours(closed_mask, cv2.RETR_EXTERNAL, cv2.CHAIN_APPROX_SIMPLE)
if contours:
    x, y, w, h = cv2.boundingRect(max(contours, key=cv2.contourArea))
    cropped_image = image.crop((x, y, x + w, y + h))
    resized_image = cropped_image.resize(size, Image.Resampling.LANCZOS)
\end{verbatim}

Ostatecznie obraz jest konwertowany do skali szarości, co redukuje wymiarowość danych i umożliwia modelowi AI skupienie się na strukturze obrazu.

\begin{verbatim}
grayscale_image = resized_image.convert("L")
\end{verbatim}

\section{Podsumowanie}

Przedstawiony algorytm przetwarzania wstępnego pozwala na skuteczne przygotowanie danych wejściowych dla modelu sztucznej inteligencji.
Automatyzuje proces identyfikacji i kadrowania obiektów zainteresowania w obrazach, co znacząco poprawia jakość danych.
Rozwiązanie zostało zaprojektowane z myślą o łatwej adaptacji do innych zastosowań wymagających podobnego przetwarzania obrazów.

