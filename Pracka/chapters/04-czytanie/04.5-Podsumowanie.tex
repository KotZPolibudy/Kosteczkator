\section{Podsumowanie}\label{sec:podsumowanie}

W procesie prac wytrenowane zostało wiele modeli, o różnej architekturze, jednak inne architektury niż przedstawiona w tym rozdziale nie dawały zadowalających wyników.
Dopiero opisana architektura oraz modele przyniosły oczekiwane rezultaty.
W rozdziale o testach~\ref{ch:testy} opisane zostały dwa wyuczone modele.
Pierwszy z nich nie korzysta w pełni z opisanych metod preprocessingu~\ref{sec:preprocessing}, a dopiero jego niezadowalające wyniki sprawiły,
że do preprocessingu dodano krok kadrujący zdjęcie do samej kości, zamiast przekazywać zdjęcie wraz z tłem.
Zmiana ta pozwoliła wyodrębnić obszar zainteresowań, co znacznie polepszyło jakość klasyfikacji przez sztuczną inteligencję.
Co więcej, drugi z modeli miał do dyspozycji znacznie zasobniejszy zbiór uczący i walidacyjny, co pozwoliło osiągnąć już niemal perfekcyjne wyniki.

% todo <- tu jest duże TODO bo bardzo nie wiem co tu opisać, a wg Julki jest to potrzebne!