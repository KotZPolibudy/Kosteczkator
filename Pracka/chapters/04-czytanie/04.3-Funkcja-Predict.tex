\section{Funkcja predict}\label{sec:funkcja-predict}

W celu rozpoznawania liczb na zdjęciach przetworzonych przez model stworzono funkcję \texttt{predict\_number},
która przekształca surowy obraz pobrany wcześniej z kamery do odpowiedniego formatu i zwraca przewidywaną klasę.
Funkcja działa w następujących krokach:
\begin{itemize}
    \item Wczytanie i przetworzenie obrazu
    \item Klasyfikacja obrazu (odczytanie wyniku)
    \item Interpretacja wyniku
\end{itemize}
Opisanych w następujących podrozdziałach.

\subsection{Wczytanie i przetworzenie obrazu}\label{subsec:wczytanie-i-przetworzenie-obrazu}

W przypadku funkcji dokonującej predykcji na pliku, obraz należy najpierw wczytać.
Krok ten jest pomijany, gdy do funkcji zostanie przekazany jako parametr obraz wczytany wcześniej.
Obraz wejściowy następnie jest poddawany takim samym transformacjom, jak przy obrazach na których trenowany był model,
a więc skalowany jest na rozmiar zgodny z wymaganiami modelu ($64 \times 64$ pikseli), zmieniany w skalę szarości, oraz normalizujemy.

\subsection{Klasyfikacja obrazu}\label{subsec:klasyfikacja-obrazu}

Przygotowany obraz jest przekazywany do modelu w celu uzyskania wyników klasyfikacji:

\begin{verbatim}
prediction = model.predict(obraz_wejściowy)
klasyfikacja = np.argmax(prediction, axis=1)
\end{verbatim}

Funkcja \texttt{np.argmax} zwraca indeks klasy o najwyższym prawdopodobieństwie.

\subsection{Interpretacja wyniku}\label{subsec:interpretacja-wyniku}

Na podstawie indeksu przewidywanej klasy wybierana jest odpowiadająca jej etykieta (od 1 do 8).
Wynik funkcji jest tą etykietą odpowiadająca numerowi na kości,
co umożliwia dalsze wykorzystanie tego wyniku w systemie rozpoznawania liczb.

W dalszym przetwarzaniu, aby przejść z cyfr w systemie dziesiętnym na zapis bitowy, dokonywana jest jedna nieoczywista transformacja.
Zakres $[1, 8]$ da się zapisać za pomocą trzech bitów, wykorzystując go w pełni (co było również przesłanką w wyborze kości o dokładnie takiej ilości ścianek).
Zdecydowano się interpretować ściankę 8 jako cyfrę 0, wobec czego, w transformacji na zapis bitowy,
8 jest interpretowane jako 000, zamiast 1000 jak mogłoby się intuicyjnie wydawać.

Finalnie, zaetykietowany wynik przetwarzany jest na ciąg trzech bitów, odpowiadających etykiecie klasy.
