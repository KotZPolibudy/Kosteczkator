\section{Funkcja predict}

W celu rozpoznawania liczb na zdjęciach przetworzonych przez model stworzono funkcję \texttt{predict\_number},
która przekształca obraz do odpowiedniego formatu i zwraca przewidywaną klasę.
Funkcja działa w następujących krokach:

\subsection{Wczytanie i przetworzenie obrazu}

W przypadku funkcji dokonującej predykcji na pliku, obraz należy najpierw wczytać. Krok ten jest pomijany gdy do funkcji zostanie przekazany jako parametr obraz wczytany wcześniej.
#TODO (to jeszcze nie istnieje, ale będzie istnieć w finalnej częście gdzie przetwarzanie będzie bardziej "potokowe" niźli wstadowe jak teraz, jeśli tak to można określić)
Obraz wejściowy wczytywany jest w trybie skali szarości i zmieniany na rozmiar zgodny z wymaganiami modelu ($64 \times 64$ pikseli):

\begin{verbatim}
img = image.load_img(img_path, target_size=(64, 64), color_mode='grayscale')
img_array = image.img_to_array(img) / 255.0
img_array = np.expand_dims(img_array, axis=0)
\end{verbatim}

Obraz jest normalizowany do zakresu $[0, 1]$, co pozwala na skuteczniejsze działanie sieci neuronowej.

\subsection{Predykcja klasy}

Przygotowany obraz jest przekazywany do modelu w celu uzyskania wyników predykcji:

\begin{verbatim}
prediction = model.predict(img_array)
predicted_class = np.argmax(prediction, axis=1)
\end{verbatim}

Funkcja \texttt{np.argmax} zwraca indeks klasy o najwyższym prawdopodobieństwie.

\subsection{Interpretacja wyniku}

Na podstawie indeksu przewidywanej klasy wybierana jest odpowiadająca jej etykieta (od 1 do 8):

\begin{verbatim}
class_names = ['1', '2', '3', '4', '5', '6', '7', '8']
predicted_label = class_names[predicted_class[0]]
return predicted_label
\end{verbatim}

Wynik funkcji to etykieta odpowiadająca numerowi na kości, co umożliwia dalsze wykorzystanie tego wyniku w systemie rozpoznawania liczb.
