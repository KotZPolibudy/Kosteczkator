\usepackage[utf8]{inputenc}
\usepackage[T1]{fontenc}
\usepackage{blindtext}

\chapter{Porażki i wnioski}\label{ch:porazki-i-wnioski}
%%To powinien być rozdział storyofmajlife
Praca inżynierska jest niemałym przedsięwzięciem, wobec czego musiały pojawić się też pewne trudności.
Niektóre z nich widzieliśmy już z daleka na wczesnych etapach planowania, a inne zaskoczyły nas, gdy pojawiały się w najróżniejszych momentach.
Na szczęście jesteśmy w stanie poradzić sobie z każdą taką trudnością, wyciągając odpowiednie wnioski. \n


\section{Faza planowania i pomysłów}\label{sec:faza-planowania-i-pomysow}

Początkowy plan zakładał rzucanie kostką, sprawdzanie wyniku, zapisywanie go i przekazywanie gdzieś dalej.
Prosty.
Aż za prosty.
Jak odczytać wynik?
Znaleźć kostkę, użyć dostępnego ocr.
%at this point, to nie wiem jak to napisać, julka pomocy bo ja dam tylko storyofmajlajf
A jak rzucić jeszcze raz?
Jakoś trzeba ją chwycić...
Może nie wypuszczać, albo upewnić się że zawsze wyląduje w tym samym miejscu?
Ale że jak wieża do kości? to jak przeładować?
No to może jednak nie zmieniać miejsca kostki, zamiast wieży do kości użyć kubka.
Zaprojektujmy własny kubek i nim obracajmy!
Problem2: kostka się nie obraca w kubku, tylko klei do ścianki
Solution2.1: Inny kubek
Nie, inne designy kubków nie zadziałały
Solution2.2: Przechylimy kubek
W ten sposób powstaje wariant betoniarka
Problem z wariantem betoniarka?
No jest dość powolny wolny..
A gdyby tak dać tam śmigiełko do środka, zamiast kręcić kubkiem?
- Brawo, zrobiliśmy blender.
Blender działa trochę lepiej, bo szybciej, ale ciekawe czy bardziej losowo?
Jak starczy czasu to sprawdzimy


\section{Hardware}\label{sec:hardware}

Tutaj problemy hardware???


\section{Czytanie wyniku}\label{sec:czytanie-wyniku}

Ciemne zdjęcia, polegamy na zewnętrznym oświetleniu.
Solution: dodać diody
Problem? zasilanie? ale to hardware.

Ok, mamy to, ale, jak tu czytać? gotowe modele?
no więc tesseract widzi "śmieci". Dziwne słowa.
Filtr na liczby i do przodu.
Ok, miliony.. hmmm...
szukaj cyfry? - nie ma cyfry, ale jest dłuuuga

Solution: preprocessing.
kontrast
nie działa, nadal słabo
własny model?
nooo, średnia skuteczność
Ok, to wróćmy do tesseracta, więcej preprocesing.
Szukanie kości na zdj i przycinanie, jak?
użyć gotowych modeli object detection np yolo8v
no więc jest 1 apple, 1 donut, no dice
Ok, może jednak nie chcę rozpoznawać, tylko szukać kostek?
po kolorach?
Po kolorach działa wycinanie, ale tesseract z wyciętego nadal widzi śmieci
No to może jednak wyrównac rozmiary tych wyciętych zdj.... no i zrobić jednak własny model
No i teraz jesteśmy tutaj.

A i poszliśmy dalej, a wesoły cyrk wciąż gra!
Własny model. Ta, dobra, to co, w czym, ile warstw, po ile neuronów?
No dobra, tutorial w mądrej książce mówi keras.
O, mamy przykładowy kod.
klik, dobra, jest, działa.

słabo, ale działa
Ale w połączeniu tego, z przycinaniem, rzucaniem do grayscale itd ogólny pre-processing iii

i nawet działa
znaczy dopiero teraz, kiedy nie uczy się na 300 zdjęciach, tylko na 2800..

i jesteśmy tu... :D

oj i dalej, a ile tu do opisania tych problemów jeszcze jest, hahah



