\chapter{Komunikacja i zastosowanie}\label{ch:komunikacja-i-zastosowanie}
%TODO - ten rozdział w większości należy już do Kuby(P)




% Po odczytaniu wartości z kości przez model następuje zagregowanie bitowych reprezentacji
% wyników w kolejkę FIFO z której grupami po 8 bitów wysyłane są kolejne generowane ciągi
%, bla bla bla
%a właściwie po co grupować je po 8?
%wsm nie wiem, Kuba wytłumacz mi, proszę.
%Dlaczego nie możemy wypluwać tak po 3 tak jak generujemy, tylko akurat po 8

%bo mamy k8? XDD
%(hehe)

%nie no, chodzi o (int) prawda? czy jakaś łatwiejsza komunikacja, albo coś?

% Dane wysyłają się w bajtach a to 8 bitów, teoretycznie moglibyśmy wysyłać 
% po prostu kążdą wygenerowaną liczbę

