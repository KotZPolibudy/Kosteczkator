\section{Przetwarzanie wyniku}

Finalnie, zaetykietowany wynik przetwarzany jest na ciąg trzech bitów, odpowiadających etykiecie klasy.
Wyjątkiem w tym przetwarzaniu jest etykieta 8, gdyż w zapisie binarnym zawierała by 4 bity, zamiast oczekiwanych trzech, jednak biorąc 3 najmniej znaczące bity (000) jesteśmy w stanie zapewnić unikatowe kodowanie każdemu wynikowi.
W ten sposób, klasy 1-7 generują bity będące ich dosłowną reprezentacją binarną, a 8 jest równoznaczne wyrzuceniu "zera" na kostce.



%TODO - ten rozdział w większości należy już do Kuby(P), ale zacząłem go jeszcze ja. Chyba że uważacie że ten szczegół należy zawrzeć jeszcze przy modelu, lub wprost tam zaetykietować klasę 8 jako 0 i przekazywać już 0, zamiast tłumaczyć 8 na 0 w tym miejscu tutaj.

