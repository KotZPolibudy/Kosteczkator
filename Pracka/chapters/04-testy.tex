\chapter{Testy}

\section{Użyte testy}

\textit{Najszerzej stosowanym w świecie narzędziem weryfikacji generatorów
(z racji dominowania standardów amerykańskich w dziedzinie ochrony informacji) jest zestaw testów podany przez normę 
amerykańską FIPS-140-2, dotyczącą bezpieczeństwa modułów kryptograficznych. W badaniu generatorów ciągów bitów losowych 
norma przewiduje cztery testy istotności. Każdy z nich przeprowadzany jest dla ciągu długości 20000 bitów}.
[Kotulski, 2001, s. 60]

\subsection{Test monobitowy}
Test monobitowy bada proporcję między liczbą zer a liczbą jedynek w otrzymanym ciągu bitów. Dla wszystkich 
wygenerowanych bitów zliczono liczbę jedynek w ciągu (\textit{X}). Nie ma podstaw do odrzucenia ciągu, gdy:
\begin{displaymath}
    9725 < X < 10275
\end{displaymath}

\subsection{Test serii}
Celem testu serii jest zliczenie tak zwanych \textit{serii}, czyli nieprzerwanych ciągów takich samych bitów. Test serii
sprawdza, czy ilość serii każdej długości jest zgodna z oczekiwanymi wartościami. Test serii bada, czy zmiany między 
wartościami bitów nie są zbyt częste bądź zbyt rzadkie. Nie ma podstaw do odrzucenia badanego ciągu, gdy:
\begin{table}
    \centering
    \begin{tabular}{|c|c|} 
        \hline
        Długość serii & Przedział \\
        \hline
        1 & 2343 - 2657 \\
        \hline
        2 & 1135 - 1365 \\
        \hline
        3 & 542 - 708 \\
        \hline
        4 & 251 - 373 \\
        \hline
        5 & 111 - 201 \\
        \hline
        6 i więcej & 111 - 201 \\
        \hline  
    \end{tabular} 
    \caption{Oczekiwane liczby wystąpień serii}
\end{table}   
\subsection{Test długich serii}
Test długich serii polega na sprawdzeniu, czy w testowanym ciągu bitów nie ma zbyt wielu występujących pod rząd takich 
samych bitów. Ciągi 20 000 bitów nie powinny zawierać serii dłuższych niż 25 bitów.

\subsection{Test pokerowy}
Test pokerowy wykorzystuje statystykę chi-kwadrat. Polega na podziale badanego ciągu na segmenty 4-bitowe i zliczeniu
liczby wystąpień każdej możliwej z szesnastu kombinacji \begin{math}s_i\end{math}. Uznaje się, że nie ma podstaw do 
odrzucenia ciągu, gdy:
\begin{displaymath}
    2,17 < X < 46,17
\end{displaymath}
Gdzie:
\begin{displaymath}
    X = \frac{16}{5000} \sum^{15}_{i=0} s_i^2 - 5000
\end{displaymath}







\subsection{Test chi-kwadrat}
Aby przeprowadzić ten test, dla wszystkich \begin{math} n \end{math} rzutów zliczono, ile razy wypadła każda z 
\begin{math} k \end{math} ścian kości ośmiościennej (\begin{math}O_i\end{math}). Wyniki te porównano z oczekiwaną
liczbą wyrzucenia każdej ze ścianek \begin{math}E_i = \frac{n}{k}\end{math}.
[zdefiniować hipotezę]


\begin{displaymath}
    \chi^2 = \sum_{i=1}^{k} \left( \frac{O_i - E_i}{E_i} \right)^2
\end{displaymath}
Ponieważ do generowania liczb losowych użyto kości ośmiościennej, to \begin{math} k \end{math} jest równe 8, co daje wzór:
\begin{displaymath}
    \chi^2 = \sum^{8}_{i=1} \left( \frac{O_i - E_i}{E_i} \right)^2
\end{displaymath}
\textit{Poziomem istotności (\begin{math} \alpha \end{math}) nazywa się prawdopodobieństwo popełnienia błędu pierwszego rzędu. Najczęściej przyjmuje się
poziom istotności \begin{math} \alpha = 0{,}05\end{math}.} [Koziarska i Metelski, 2016, s. 82]
Stopień swobody przy \begin{math} k = 8 \end{math} równa się:
\begin{displaymath}
    k - 1 = 8 - 1 = 7
\end{displaymath}
Zatem dla otrzymanego stopnia swobody oraz założonego stopnia istotności, wartość krytyczna wynosi \begin{math} 14,0671 \end{math}.
Jeśli otrzymana wartość będzie mniejsza od wartości krtycznej, nie ma podstaw do odrzucenia hipotezy zerowej.
