\chapter{Testy}

Przez wzgląd na brak oprogramowania, nie da się przeprowadzić typowej oceny jakości projektu. Zamiast tego, postanowiono 
sprawadzić losowość odczytanych wyników rzutu kością dla obu wariantów robota, a następnie porównać je między sobą oraz
z losowością wbudowanego generatora pseudolosowego programu Python.

Do oceny losowości otrzymanych wyników użyto:
\begin{itemize}
    \item test chi-kwadrat,
    \item test pojedynczych bitów,
    \item test serii,
    \item test długiej serii,
    \item test pokerowy
    \item procentową wartość entropii.
\end{itemize}

Dla otrzymania wystarczającej ilości danych, należało wygenerować 20 000 bitów. Ponieważ robot używa kości ośmiościennej,
dającej wynik w postaci dokładnie 3 bitów, należało wykonać 6 667 rzutów.

\section{Użyte wzory}

\subsection{Test chi-kwadrat}
Aby przeprowadzić ten test, dla wszystkich \begin{math} n \end{math} rzutów zliczono, ile razy wypadła każda z 
\begin{math} k \end{math} ścian kości ośmiościennej (\begin{math}O_i\end{math}). Wyniki te porównano z oczekiwaną
liczbą wyrzucenia każdej ze ścianek \begin{math}E_i = \frac{n}{k}\end{math}.


\begin{displaymath}
    \chi^2 = \sum_{i=1}^{k} \left( \frac{O_i - E_i}{E_i} \right)^2
\end{displaymath}
Ponieważ do generowania liczb losowych użyto kości ośmiościennej, to \begin{math} k \end{math} jest równe 8, co daje wzór:
\begin{displaymath}
    \chi^2 = \sum^{8}_{i=1} \left( \frac{O_i - E_i}{E_i} \right)^2
\end{displaymath}
Dla przeprowadzonych testów wybrano poziom istotności \begin{math} p = 0,5 \end{math}, natomiast stopień swobody 
przy \begin{math} k = 8 \end{math} równa się:
\begin{displaymath}
    k - 1 = 8 - 1 = 7
\end{displaymath}
Zatem dla otrzymanego stopnia swobody oraz założonego stopnia istotności, wartość krytyczna wynosi \begin{math} 14,0671 \end{math}.
Test jest zakończony sukcesem, jeśli otrzymana wartość będzie mniejsza od wartości krtycznej.

\subsection{Entropia}
\begin{displaymath}
    H \left(X\right) = -\sum^k_{i=1} p \left(x_i\right) log_2 p \left(x_i\right)
\end{displaymath}
Ponieważ wartość entropii jest wyznaczana dla bitów, a każdy rzut na kości ośmiościennej daje wynik w 3 bitach, 
wartość entropii przeprowadzonego testu może wynosić maksymalnie 3.

\subsection{Test pojedynczych bitów}
Dla wszystkich wygenerowanych bitów zliczono, ile razy wystąpiły 0 (\begin{math}n_0\end{math}) i ile razy wystąpiły 1 
(\begin{math}n_1\end{math}). Test jest zakończony sukcesem, jeśli każda z tych wartości spełnia zależność:
\begin{displaymath}
    9725 < n_0 < 10275
\end{displaymath}
\begin{displaymath}
    9725 < n_1 < 10275
\end{displaymath}

\subsection{Test serii}
Jednocześnie dla ciągu wygenerowanych bitów zliczono tzw. serie, czyli ile razy bez przerw występuje podciąg zer lub 
jedynek o długościach od jednego do pięciu takiego samego bitu pod rząd. Dłuższe serie zliczono do jednej kategorii.
Aby test został zakończony sukcesem, liczba serii danych długości dla zarówno zer jak i jedynek musi spełniać zależność
przedstawioną w poniższej tabeli. 
\begin{center}
    \begin{tabular}{|c|c|} 
        \hline
        Długość serii & Przedział \\
        \hline
        1 & 2315 - 2685 \\
        \hline
        2 & 1114 - 1386 \\
        \hline
        3 & 527 - 723 \\
        \hline
        4 & 240 - 384 \\
        \hline
        5 & 103 - 209 \\
        \hline
        6 i więcej & 103 - 209 \\
        \hline
    \end{tabular}        
\end{center}
\subsection{Test najdłuższej serii}
Przy zliczaniu bitów dla testu serii, jednocześnie szukano najdłuższego podciągu samych zer i najdłuższego podciągu 
samych jedynek. Test długiej serii jest zakończony sukcesem, jeśli żaden z tych podciągów nie zawiera więcej niż 25 bitów.

\subsection{Test pokerowy}
Do przeprowadzenia testu pokerowego podzielono wygenerowany ciąg na czwórki bitów, a następnie zliczono, ile razy 
wystąpiła każda z szesnastu kombinacji \begin{math}s_i\end{math}. Test jest zakończony sukcesem, jeśli spełniona jest
poniższa nierówność:
\begin{displaymath}
    2,16 < X < 46,17
\end{displaymath}
Gdzie:
\begin{displaymath}
    X = \frac{16}{5000} \sum^{15}_{i=0} s_i^2 - 5000
\end{displaymath}


\section{Testy poszczególnych wariantów}
Coś tam coś tam, mamy dwa warianty robota to możemy je porównać. (jeśli będzie czas na testowanie betoniarki)
\subsection{Wariant 1 - \glqq Betoniarka\grqq}
\subsection{Wariant 2 - \glqq Blender\grqq}

