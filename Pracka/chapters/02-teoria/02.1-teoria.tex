\chapter{Fragment teoretyczny}

---- TODO ---- \\

Co i dlaczego tak w sumie budujemy? \\

Jedną z najważniejszych decyzji w trakcie realizacji projektu był wybór rodzaju kości, jaką miał odczytywać 
i interpretować model sztucznej inteligencji. Jako najistotniejsze kryterium wyznaczono liczbę ścian będącą 
potęgą liczby dwa - w celu łatwej interpretacji wyniku w notacji binarnej, stosowanej powszechnie w kryptografii.
W następnej kolejności szukano kompromisu pomiędzy łatwością w odczycie ścianek a generowaniem jak największej liczby
bitów jednym rzutem - czyli kości o jak największej liczbie ścianek. 
\par
Z dostępnych powszechnie na rynku kości tylko cztero-, ośmio- i szesnastościenne spełniają pierwszy z wymienionych
wyżej kryteriów. Kość czterościenną odrzucono ze względu na jej kształt ostrosłupa foremnego o podstawie trójkąta, 
na której liczby zapisywane są na rogach a nie ściankach. Kość szesnanstościenną również odrzucono ze względu na
mały rozmiar ścianek, który sprawia, że na zdjęciu robionym idealnie nad kością byłoby widać kilka ścianek na raz.
Z pozostałych opcji tylko na kości ośmiościennej widać z góry dokładnie jedną ściankę, zatem to ten rodzaj kości
został wybrany do projektu.