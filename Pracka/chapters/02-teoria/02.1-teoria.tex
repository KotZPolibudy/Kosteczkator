\chapter{Obecnie stosowane generatory liczb losowych}\label{ch:przeglad-rynku}

Współczesne systemy kryptograficzne oraz aplikacje wymagają generowania wysokiej jakości liczb losowych,
które muszą charakteryzować się wysoką jakością i odpornością na przewidywalność.
Najlepiej to założenie spełniają generatory liczb prawdziwie losowych, czyli te oparte na zjawiskach fizycznych.
Jest to spowodowane tym, że kiedy w układach fizycznych bierze się pod uwagę wiele czynników, to nawet najmniejsze zmiany
w którymś z tych czynników sprawiają, że wynik jest odmienny.
Z tego powodu nie można przewidzieć wyniku działania takiego układu, a uzyskane wyniki są losowe~\cite{chaos}.

Jednak powszechnie używane są generatory pseudolosowe (PRNG, z ang. \textit{Pseudorandom Number Generator}).
PRNG są całkowicie deterministyczne i zależą od zadanej im wartości początkowej zwanej ziarnem (ang. \textit{seed}).
TRNG (z ang. \textit{True Random Number Generator}) są szczególnie istotne w kontekście aplikacji wymagających silnej 
ochrony danych, takich jak systemy kryptograficzne, podpisy cyfrowe, oraz w generowaniu kluczy szyfrujących,
gdyż nie da się przewidzieć w skuteczny sposób generowanych przez nie wartości.

\section{Obecnie stosowane rozwiązania}\label{sec:obecnie-stosowane-rozwiazania}

% todo opisać to lepiej, może Kuba (Yoshida) o linuxowych?
Obecnie komputery osobiste wykorzystują generatory liczb losowych (RNG, z ang. \textit{Random Number Generator})
wykorzystujące dostępne im źródła entropii dostarczane przez użytkownika, takie jak ruchy myszy lub naciśnięcia przycisków.
W nowoczesnych procesorach stosuje się też liczby generowane przez dedykowane moduły takie jak Intel DRNG za pomocą \textit{Intel RDSEED} i \textit{RDRAND}~\cite{IntelRD}.
System Linux dostarcza również interfejsy \textit{/dev/random} raz \textit{/dev/urandom} zapewniające liczby losowe.



