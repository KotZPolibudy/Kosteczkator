\chapter{Fragment teoretyczny}

Współczesne systemy kryptograficzne oraz aplikacje wymagają generowania liczb losowych, które muszą charakteryzować się wysoką jakością i odpornością na przewidywalność. Najlepiej spełniają do generatory liczb losowych oparte na fizycznych zjawiskach, zwane TRNG (True Random Number Generator). TRNG są szczególnie istotne w kontekście aplikacji wymagających silnej ochrony danych, takich jak systemy kryptograficzne, podpisy cyfrowe, oraz w generowaniu kluczy szyfrujących.

\section{Zalety dostępnych rozwiązać TRNG}
AAAAAAAAAAA, AAAAAAAAAA AAA AAAAAAAAAAAA

\section{Wady dostępnych rozwiązań TRNG}

Mimo że Generatory Liczb Losowych Oparte na Zjawiskach Fizycznych (TRNG) oferują wysoki poziom bezpieczeństwa i losowości, istnieje kilka istotnych wad i ograniczeń związanych z ich implementacją i użytkowaniem. Wśród głównych problemów, które mogą wpływać na wydajność oraz niezawodność TRNG, wyróżnia się następujące aspekty:

\subsection{Wydajność i szybkość generowania liczb losowych}

Wydajność TRNG jest często niższa niż w przypadku Generujących Liczby Losowe Oparte na Algorytmach (PRNG). Generowanie liczb losowych za pomocą zjawisk fizycznych, takich jak szum termiczny czy fluktuacje kwantowe, może być procesem czasochłonnym, szczególnie w systemach wymagających dużych ilości losowych liczb w krótkim czasie. W wyniku tego, TRNG mogą nie spełniać wymagań wydajnościowych w aplikacjach o dużym zapotrzebowaniu na losowość, takich jak systemy kryptograficzne o bardzo wysokiej częstotliwości operacji.

\subsection{Koszty implementacji}

Urządzenia TRNG, zwłaszcza te oparte na zaawansowanych technologiach, takich jak fotonika kwantowa czy detekcja szumów kwantowych, mogą wiązać się z wysokimi kosztami produkcji oraz utrzymania. Z tego powodu, TRNG są często droższe w porównaniu do bardziej ekonomicznych rozwiązań opartych na algorytmach deterministycznych (PRNG), które wystarczają do wielu zastosowań, gdzie wysoka jakość losowości nie jest kluczowym wymaganiem.

\subsection{Stabilność i jakość generowanych liczb losowych}

Choć TRNG są uznawane za bezpieczne, ich jakość może być wpływana przez różne czynniki zewnętrzne, takie jak temperatura, zakłócenia elektromagnetyczne czy inne zmiany środowiskowe. W wyniku tego, generowane liczby losowe mogą wykazywać pewne niskiej jakości właściwości, co wymaga zastosowania dodatkowych mechanizmów, takich jak procesy post-przetwarzania, aby zapewnić ich odpowiednią losowość. Nawet małe zakłócenia w systemie mogą prowadzić do wzorców, które mogą zostać wykorzystane w atakach kryptograficznych.

\subsection{Skomplikowana kalibracja i konserwacja}

Urządzenia TRNG, szczególnie te, które wykorzystują skomplikowane zjawiska fizyczne, wymagają starannej kalibracji i ciągłego monitorowania, aby zapewnić ich prawidłowe funkcjonowanie. W wielu przypadkach konieczne jest stosowanie systemów nadzoru, które monitorują jakość generowanych liczb losowych w czasie rzeczywistym. Ponadto, wymogi dotyczące utrzymania odpowiednich warunków pracy, takich jak stabilna temperatura czy brak zakłóceń elektromagnetycznych, mogą stanowić dodatkową przeszkodę w ich szerokim zastosowaniu.

\subsection{Złożoność integracji z istniejącymi systemami}

Integracja TRNG z już działającymi systemami, zwłaszcza w kontekście urządzeń wbudowanych lub systemów o dużych wymaganiach obliczeniowych, może wiązać się z wieloma trudnościami. Często konieczne jest dostosowanie sprzętu lub oprogramowania w celu zapewnienia kompatybilności i pełnej funkcjonalności. Dodatkowo, ze względu na fizyczną naturę tych urządzeń, integracja z innymi komponentami może prowadzić do wzrostu kosztów oraz złożoności całego systemu.

\subsection{Ograniczona dostępność i skalowalność}

Choć rynek TRNG rośnie, nadal jest on stosunkowo niszowy w porównaniu do bardziej powszechnych rozwiązań opartych na PRNG. Ograniczona dostępność wyspecjalizowanych urządzeń TRNG, szczególnie tych, które oferują wysoką jakość generowanych liczb losowych, sprawia, że ich wdrożenie jest trudniejsze, zwłaszcza w przypadkach wymagających masowej produkcji. Ponadto, nie wszystkie rozwiązania są skalowalne, co może być problemem w przypadku aplikacji wymagających elastyczności i łatwego dostosowywania wydajności do rosnących potrzeb.

\subsection{Zagadnienia związane z bezpieczeństwem}

Chociaż TRNG zapewniają wyższy poziom bezpieczeństwa niż PRNG, nie są one całkowicie odporne na ataki. Ataki fizyczne, takie jak manipulacje w obrębie urządzenia lub przechwytywanie sygnałów z jego elementów, mogą prowadzić do kompromitacji jakości liczb losowych. Ponadto, w przypadku rozwiązań opartych na technologii chmurowej, takich jak te oferowane przez Cloudflare, istnieje ryzyko ataków związanych z przechwytywaniem lub manipulowaniem danymi w trakcie transmisji, co może wpływać na bezpieczeństwo generowanych liczb losowych.

\subsection{Podsumowanie wad TRNG}

Chociaż TRNG oferują niezrównaną jakość losowości w porównaniu do rozwiązań opartych na algorytmach, posiadają również szereg wad, które mogą ograniczać ich szerokie zastosowanie. Należy do nich niska wydajność, wysokie koszty implementacji, problemy ze stabilnością generowanych liczb losowych, złożoność integracji z systemami oraz ryzyko związane z bezpieczeństwem. W miarę jak technologia będzie się rozwijać, możliwe jest, że te ograniczenia zostaną przezwyciężone, jednak obecnie stanowią one istotne wyzwanie dla szerokiego przyjęcia TRNG w różnych aplikacjach.

\section{Podsumowanie dostępnych na rynku TRNG}