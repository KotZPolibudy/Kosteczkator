\section{Dostępne na rynku TRNG}\label{sec:dostepne-na-rynku-trng}

tu ogólnie, to co jest obecnie wypisane na tym githubie którego Kuba podrzucił na issues

\subsection{Podobne rozwiązania}

Szukając dostępnych na rynku TRNG, szczególną uwagę poświęcono również tym najbardziej zbliżonym do realizowanego projektu.
Wśród znalezionych rozwiązań zaledwie dwa inspirowały się fizycznym rzutem kością. 

Pierwszym ze znalezionych rozwiązań był projekt \textit{The Smart Dice Cup}, zrealizowany przez Carstena Magerkurth, 
Timo'a Engelke i Carstena Röcker \cite{SmartDice}. Jest on inspirowany tradycyjnymi kubkami do gry w kości, przypomina
je w swojej konstrukcji i sposobie działania. Urządzeniem należy potrząsnąć, aby uzyskać losową 
liczbę. Wynik takiego „rzutu” pokazywany jest na wyświetlaczu LED, zamontowanym na „pokrywce” urządzenia.

Podstawową różnicą między realizowanym projektem a \textit{The Smart Dice Cup} jest brak wykorzystania prawdziwych 
kości w drugim rozwiązaniu. Zamiast tego wynik jest generowany na podstawie zmiany prędkości, z jaką potrząsane jest 
urządzenie.

Drugim rozwiązaniem jest \textit{Dice thrown by cup and machine in PK tests} autorstwa J. B. Rhine \cite{betoniarka43}. 
Jest on najbardziej zbliżony do przedstawionego w poniższej pracy projektu, ponieważ również wykonuje wprawienie kości w 
ruch za pomocą obracającego się mechanicznie pojemnika. Z racji na dostępne technologie, z którymi pracował J. B. Rhine,
odczytywanie wyników musiało odbywać się ręcznie, a nie za pomocą małej kamery i sztucznej inteligencji. Jednakże,
konstrukcja urządzenia jest bardzo podobna do zaprezentowanego w sekcji \ref{subsec:prototypowanie} \textit{betoniarki}.