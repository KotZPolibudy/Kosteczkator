\chapter{Testy}

\section{Użyte testy}
\label{testyOpis}

\textit{Najszerzej stosowanym w świecie narzędziem weryfikacji generatorów
(z racji dominowania standardów amerykańskich w dziedzinie ochrony informacji) jest zestaw testów podany przez normę 
amerykańską FIPS-140-2, dotyczącą bezpieczeństwa modułów kryptograficznych. W badaniu generatorów ciągów bitów losowych 
norma przewiduje cztery testy istotności. Każdy z nich przeprowadzany jest dla ciągu długości 20000 bitów (w przypadku 
wykorzystywania dłuższych ciągów przebadane muszą być kolejne podciągi o tej długości). Poziom istotności tych testów, 
czyli prawdopodobieństwo odrzucenia ciągu mogącego pochodzić z prawidłowego źródła bitów, jest równy 0,0001.}
\cite{Kotulski2001} \par
W punktach \ref{monbitOpis} - \ref{pokerowyOpis} zostały opisane testy wskazane w owej normie. Przedsawione w nich stałe
są zgodnee z nomrami przedstawionymi w FIPS 140-2 \cite{NIST2001}. Zostały one zaimplementowane w języku Python.
\par
Do przeprowadzenia testów rozważano również wykorzystanie biblioteki \textit{TestU01}. Niestety, do zastosowania nawet 
najmniejszego z zaproponowanych w niej testów potrzeba przynajmniej \begin{math} 10^6 \end{math} bitów. Przez 
wzgląd na ograniczenia czasowe oraz ryzyko nadmiernej eksploatacji robota przy generowaniu tak dużej ilości danych, 
zrezygnowano z wykorzystania tej biblioteki.


\subsection{Test chi-kwadrat}
Sformuowano następującą hipotezę:
\par \begin{math} H_0 \end{math}: Rozkład prawdopodobieństwa wyników jest równomierny.
\par \begin{math} H_1 \end{math}: Rozkład prawdopodobieństwa wyników nie jest równomierny.
\par Aby zweryfikować hipotezę zerową, dla wszystkich \begin{math} n \end{math} rzutów odczytano, ile razy wypadła każda z 
\begin{math} k \end{math} ścian kości ośmiościennej (\begin{math}O_i\end{math}). Wyniki te porównano z oczekiwaną
liczbą wyrzucenia każdej ze ścianek \begin{math}E_i = \frac{n}{k}\end{math}. 

\begin{displaymath}
    \chi^2 = \sum_{i=1}^{k} \left( \frac{O_i - E_i}{E_i} \right)^2
\end{displaymath}
Ponieważ do generowania liczb losowych użyto kości ośmiościennej, to \begin{math} k \end{math} jest równe 8, co daje wzór:
\begin{displaymath}
    \chi^2 = \sum^{8}_{i=1} \left( \frac{O_i - E_i}{E_i} \right)^2
\end{displaymath}
\par Stopień swobody przy \begin{math} k = 8 \end{math} równa się:
\begin{displaymath}
    k - 1 = 8 - 1 = 7
\end{displaymath}
Zatem dla otrzymanego stopnia swobody oraz założonego stopnia istotności, wartość krytyczna wynosi \begin{math} 
14{,}0671 \end{math}. Jeśli otrzymana wartość będzie mniejsza od wartości krtycznej, nie ma podstaw do odrzucenia 
hipotezy zerowej.


\subsection{Test monobitowy}
\label{monbitOpis}
Test monobitowy bada proporcję między liczbą zer a liczbą jedynek w otrzymanym ciągu bitów. Dla wszystkich 
wygenerowanych bitów zliczono liczbę jedynek w ciągu (\textit{X}). Oczekiwana liczba jedynek w ciągu wynosi:
\begin{displaymath}
    9725 < X < 10275
\end{displaymath}
\par \begin{math} H_0 \end{math}: Proporcja zer i jedynek jest zgodna z oczekiwaną.
\par \begin{math} H_1 \end{math}: Proporcja zer i jedynek nie jest zgodna z oczekiwaną.

\subsection{Test serii}
Celem testu serii jest zliczenie tak zwanych \textit{serii}, czyli nieprzerwanych ciągów takich samych bitów. Test serii
sprawdza, czy ilość serii każdej długości jest zgodna z oczekiwanymi wartościami. Test serii bada, czy zmiany między 
wartościami bitów nie są zbyt częste bądź zbyt rzadkie. Oczekiwane rozkłady wystąpień poszczególnych serii przedstawiono
w tabeli \ref{serieOczekiwane}.
\par \begin{math} H_0 \end{math}: Rozkład wystąpień serii bitów jest zgodny z przyjętym rozkładem.
\par \begin{math} H_1 \end{math}: Rozkład wystąpień serii bitów nie jest zgodny z przyjętym rozkładem.
\begin{table}[h]
    \centering
    \caption{Oczekiwane liczby wystąpień serii}
    \label{serieOczekiwane}
    \begin{tabular}{|c|c|} 
        \hline
        Długość serii & Przedział \\
        \hline
        1 & 2343 - 2657 \\
        \hline
        2 & 1135 - 1365 \\
        \hline
        3 & 542 - 708 \\
        \hline
        4 & 251 - 373 \\
        \hline
        5 & 111 - 201 \\
        \hline
        6 i więcej & 111 - 201 \\
        \hline  
    \end{tabular} 
\end{table}   
\subsection{Test długich serii}
Test długich serii polega na sprawdzeniu, czy w testowanym ciągu bitów nie ma zbyt wielu występujących pod rząd takich 
samych bitów. Ciągi 20 000 bitów nie powinny zawierać serii dłuższych niż 25 bitów. Sformuowano następujące hipotezy:
\par \begin{math} H_0 \end{math}: Wygenerowany ciąg nie zawiera długiej serii.
\par \begin{math} H_1 \end{math}: Wygenerowany ciąg zawiera długą serię.

\subsection{Test pokerowy}
\label{pokerowyOpis}
Test pokerowy wykorzystuje statystykę rozkładu chi-kwadrat. Polega na podziale badanego ciągu na segmenty 4-bitowe i 
zliczeniu liczby wystąpień każdej możliwej z szesnastu kombinacji \begin{math}s_i\end{math}. Następnie oblicza się 
statystykę testową:
\begin{displaymath}
    X = \frac{16}{5000} \sum^{15}_{i=0} s_i^2 - 5000
\end{displaymath}
Rozkład sekwencji w ciągu jest zgodny z oczekiwanym, gdy:
\begin{displaymath}
    2{,}17 < X < 46{,}17
\end{displaymath}
\par \begin{math} H_0 \end{math}: Rozkład 4-bitowych segmentów w ciągu jest zgodny z oczekiwanym.
\par \begin{math} H_1 \end{math}: Rozkład 4-bitowych segmentów w ciągu nie jest zgodny z oczekiwanym.



\section{Wyniki testów}
Przedstawione w sekcji \ref{testyOpis} testy zostały wykorzystane do sprawdzenia losowości ciągu wygenerowanego przez robota i 
odczytywnane przez oba modele sztucznej inteligencji. Ich wyniki przedstawiono w poniższej sekcji.
\subsection{Test chi-kwadrat}
\label{chiWyniki}
Dla odczytanych wyników rzutów kością przez pierwszy modelotrzymano wartość statystyki testowej chi-kwadrat równą 
246{,}079, która znacznie przekroczyła przyjętą wartość krytyczną równą \begin{math} 14{,}0671 \end{math}. 
Odrzucono hipotezę \begin{math} H_0 \end{math} i przyjęto hipotezę \begin{math} H_1 \end{math}: rozkład odczytanych 
wyników rzutów kością nie jest równy. 
Otrzymany rozkład przedstawiono w tabeli \ref{chiTabela}.
\begin{table}[h]
    \centering
    \caption{Rozkład otrzymanych wyników rzutu kością}
    \label{chiTabela}
    \begin{tabular}{|c|>{\centering\arraybackslash}m{2cm}|>{\centering\arraybackslash}m{2cm}|}
        \hline
        & \multicolumn{2}{c|}{Liczba otrzymanych rzutów} \\
        \hline
        Wynik & Model 1 & Model 2 \\
        \hline
        1 & 849 & 870 \\
        \hline
        2 & 682 & 841 \\
        \hline
        3 & 952 & 838 \\
        \hline
        4 & 831 & 819 \\
        \hline
        5 & 866 & 836 \\
        \hline
        6 & 1124 & 737 \\
        \hline  
        7 & 816 & 818 \\
        \hline 
        8 & 547 & 908 \\
        \hline 
    \end{tabular} 
\end{table}


\subsection{Test monobitowy}
\label{monbitWyniki}
Dla otrzymanego ciągu zliczono 10694 bitów o wartości 1. Jest ona większa od oczekiwanej liczby jedynek. Hipotezę zerową
odrzucono.

\subsection{Test serii}

Wyniki testów serii przedstawiono w tabeli \ref{serieTabela}. Rozkład serii w badanym ciągu nie jest zgodny z oczekiwanymi dla obu 
serii o długości 1, obu serii o długości 2, serii jedynek o długości 4 i serii zer o długości 6 lub więcej. Hipotezę
zerową odrzucono.
\begin{table}[h]
    \centering
    \caption{Liczby wystąpień serii w ciągu}
    \label{serieTabela}
    \begin{tabular}{|c|c|c|c|c|c|} 
        \hline
        \multicolumn{2}{|c|}{} & \multicolumn{2}{c|}{Model 1} & \multicolumn{2}{c|}{Model 2} \\
        \hline
        Długość serii & Przedział & Serie zer & Serie jedynek & Serie zer & Serie jedynek \\
        \hline
        1 & 2343 - 2657 & 2717 & 2303 & 2405 & 2477 \\
        \hline
        2 & 1135 - 1365 & 1367 & 1141 & 1212 & 1250 \\
        \hline
        3 & 542 - 708 & 559 & 678 & 612 & 593 \\
        \hline
        4 & 251 - 373 & 263 & 374 & 338 & 290 \\
        \hline
        5 & 111 - 201 & 113 & 158 & 162 & 155 \\
        \hline
        6 i więcej & 111 - 201 & 82 & 177 & 194 & 159 \\
        \hline  
    \end{tabular} 
\end{table}   

\subsection{Test długich serii}
W wygenerowanym ciągu zarówno dla zer, jak i dla jedynek, najdłuższa znaleziona seria zawierała 13 bitów. Jest to liczba
większa od maksymalnej przyjętej długości najdłuższego ciągu bitów, wynoszącej 25. Nie ma podstaw do odrzucenia hipotezy 
zerowej.

\subsection{Test pokerowy}
\label{pokerWyniki}
Dla testu pokerowego otrzymano statystykę testową równą 120{,}026. Jest ona większa od oczekiwanej, zatem odrzucono 
hipotezę zerową. Rozkład segmentów w ciągu przedstawiono w tabeli \ref{pokerTabela}.
\begin{table}[h]
    \centering
    \caption{Liczby wystąpień kombinacji bitów}
    \label{pokerTabela}
    \begin{tabular}{|c|c|c|} 
        \hline
        & \multicolumn{2}{c|}{Liczba wystąpień} \\
        \hline
        Kombinacja bitów & Model 1 & Model 2 \\
        \hline
        0000 & 197 & 361 \\
        \hline
        0001 & 249 & 310 \\
        \hline
        0010 & 277 & 347 \\
        \hline
        0011 & 312 & 317 \\
        \hline
        0100 & 279 & 246 \\
        \hline
        0101 & 309 & 280  \\
        \hline  
        0110 & 363 & 283 \\
        \hline  
        0111 & 338 & 294 \\
        \hline  
        1000 & 255 & 321 \\
        \hline  
        1001 & 355  & 276 \\
        \hline  
        1010 & 311 & 298 \\
        \hline  
        1011 & 355 & 322 \\
        \hline  
        1100 & 322 & 320 \\
        \hline  
        1101 & 335 & 297 \\
        \hline  
        1110 & 366 & 318 \\
        \hline  
        1111 & 377 & 310 \\
        \hline  
    \end{tabular} 
\end{table} 

\subsection{Wnioski}
Dla czterech z pięciu testów odrzucono hipotezę zerową. W wynikach testu statystyki rozkładu chi-kwadrat 
(\ref{chiWyniki}) można zauważyć, że liczba 6 jest odczytywana ponad dwa razy częściej niż liczba 8. Ma to 
bardzo duży wpływ na otrzymany ciąg bitów, ponieważ liczba 6 jest interpretowana jako 110, a liczb 8 na 000, 
co można zaobserwować w wynikach pozostałych testów. Sprawia to, że jedynki pojawiają się częściej w ciągu niż zera, 
co ukazuje już test monobitowy (\ref{chiWyniki}), gdzie liczba jedynek w ciągu jest większa niż oczekiwana. 
Podobny problem można zaobserwować w wynikach testu pokerowego (\ref{pokerWyniki}), gdzie segmenty zawierające 
trzy lub cztery bity o wartości 1 występują znacznie częściej niż segmenty zawierające więcej bitów o wartości 0. 